\begin{enumerate}[label=\thesubsection.\arabic*.,ref=\thesubsection.\theenumi]
\numberwithin{equation}{enumi}
\item
For the Wein-bridge oscillator of Fig \ref{fig:ee18btech11044_3_tikz_1}, use the expression for loop gain in Eq \ref{eq:ee18btech11044_3_1}  to find the poles of the closed-loop system. Give the expression for the pole , Q and use it to show that to locate the poles in the right half of s plane, $\frac{R_2}{R_1}$ must be selected to be greater than 2. 

\begin{figure}[!hbt]
	\begin{center}
			\resizebox{\columnwidth}{!}{\input{./figs/ee18btech11044_3_tikz_1.tex}}
	\end{center}
\caption{}
\label{fig:ee18btech11044_3_tikz_1}
\end{figure}


\item Compare the basic structure for a sinusoidal oscillator with Wein-bridge oscillator and give expressions for A and $\beta$. 

\solution
\begin{itemize}
    \item Comparring Fig \ref{fig:ee18btech11044_3_tikz_1} and Fig \ref{fig:ee18btech11044_3_tikz_2}, we get
\begin{align}
A = 1+\frac{R_2}{R_1} \\
\beta = \frac{Z_p}{Z_p + Z_s}
\end{align}
where,
\begin{align}
    Z_p = \frac{R}{RSC+1} \\
    Z_s = \frac{RSC+1}{SC}
\end{align}
\end{itemize}



\begin{figure}[!hbt]
	\begin{center}
		\resizebox{\columnwidth}{!}{\input{./figs/ee18btech11044_3_tikz_2.tex}}
	\end{center}
\caption{}
\label{fig:ee18btech11044_3_tikz_2}
\end{figure} 





\item
Give the expression for loop gain for Wein-bridge oscillator. 

\solution
\begin{align}
    L(s) = A(s) \beta(s) \\ 
    L(s) = \frac{1+\frac{R_2}{R_1}}{1 + Z_s Y_p} \\
    L(s) = \frac{1+\frac{R_2}{R_1}}{1 + (\frac{sRC + 1}{sC}) (\frac{sRC+1}{R})} \\
    L(s) = \frac{1+\frac{R_2}{R_1}}{1 + \frac{s^2R^2C^2 +sRC + sRC + 1}{sRC}} \\
    L(s) = \frac{1 + \frac{R_2}{R_1}}{3 + sCR + \frac{1}{sCR}} \label{eq:ee18btech11044_3_1}
\end{align}

\item Write the characteristic equation for Wein-bridge oscillator.

\solution
\begin{align}
    1 - L(s) = 0  \\
    1 - \frac{1 + \frac{R_2}{R_1}}{3 + sCR + \frac{1}{sCR}} = 0  \\
    3 + sRC + \frac{1}{sCR} = 1 + \frac{R_2}{R_1}  \\
    3 - 1 +sRC +\frac{1}{sRC} -\frac{R_2}{R_1} = 0  \\
    2s + s^2 RC + \frac{1}{RC} -s\frac{R_2}{R_1} = 0  \\
    s^2 RC + s(2 - \frac{R_2}{R_1}) + \frac{1}{RC} =0 \\
    s^2 + s \frac{1}{RC}(2-\frac{R_2}{R_1}) + \frac{1}{R^2C^2} = 0 \label{eq:ee18btech11044_3_2}
\end{align}

\item
Write the general expression for the characteristic equation.

\solution
\begin{align}
    s^2 + s\frac{\omega_0}{Q} + \omega_0^2 = 0 \label{eq:ee18btech11044_3_3}
\end{align}

\item State the \textbf{Barkhausen criterion} for sustained oscillations with frequency $\omega_0$.

\solution
\begin{align}
    L(j\omega_0) = A(j\omega_0) \beta(j\omega_0) = 1
\end{align}
\begin{itemize}
    \item That is, at $\omega_0$ the phase of the loop gain should be zero and the magnitude of loop gain should be 1.
    \item Only for a $\infty$ gain,system will produce a finite output for zero input. 
\end{itemize}

\item Give the definition of \textbf{Quality factor}(Q) and explain its significance.

\solution
\begin{itemize}
    \item It is a parameter of an oscillatory system expressing the relationship between stored energy and energy dissipation.
    \item The "purity" of output sine waves will be a function of the selectivity feedback network.
    \item That is, higher the value of Q for frequency selective network, the less the harmonic content of sine wave produced.
\end{itemize}
 


\item 
Compare the equations \ref{eq:ee18btech11044_3_2} and \ref{eq:ee18btech11044_3_3} and give expressions for Q and $\omega_0$

\solution
\begin{align}
    \omega_0^2 = \frac{1}{R^2C^2} \\
    \omega_0 = \frac{1}{RC}
    \frac{\omega_0}{Q} = \frac{1}{RC}(2 - \frac{R_2}{R_1}) \\
    Q = \frac{1}{(2 - \frac{R_2}{R_1})} \label{eq:ee18btech11044_3_4} \\
\end{align}
\item 
Using Eq \ref{eq:ee18btech11044_3_4} calculate the value of $\frac{R_2}{R_1}$ for which poles lie on right hand of s-plane.

\solution 

Poles lie on imaginary axis for $Q = \infty$
\begin{align}
    2 - \frac{R_2}{R_1} = 0 \\
    \frac{R_2}{R_1} = 2
\end{align}
$\therefore$ For poles to lie on right hand side of s-plane
\begin{align}
    \frac{R_2}{R_1} >2
\end{align}


\item
Verify the above calculations using a Python code.

\solution
\begin{lstlisting}
codes/ee18btech11044_3_1.py
\end{lstlisting}
\begin{itemize}
    \item This figure shows how the location of poles vary if $\frac{R_2}{R_1}$ is varied for a fixed $\omega_0$.
    \item I have varied $\frac{R_2}{R_1}$ from -10 to 10. 
\end{itemize}

\begin{figure}[!ht]
\centering
\includegraphics[width=\columnwidth]{figs/ee18btech11044_3_1.eps}
\caption{}
\end{figure}




\end{enumerate}
